\chapter{Bounding Volume Hierarchies}
  We begin our discussion of bounding volume hierarchies (or BVH) by considering
  the following example:
  \begin{exmp}
    Consider a model of a humanoid character for a fighting game. We require the
    collision detection system to be able to detect punches to each of the
    extremities, torso and head. What would be the best way of creating a
    bounding volume(s) for the model? Assume that the model is in the standard
    T-pose for your solutions.
  \end{exmp}

  Let's take a look at the bounding volumes that we already know (and let's
  bring in sphere-swept volumes as well). Suppose that we wished to use a
  sphere. Then we could either take the diameter of the sphere to be the height
  of the model, or the wingspan. If we take the height, then we waste a lot of
  space around the head, torso and legs. Also, we have no way of checking for
  intersection against individual limbs. On the other hand, if we took the
  wingspan, then chances are we loose the ability to detect collisions on the
  legs. An AABB suffers from the same problems, as do sphere-swept volumes. So
  how can we solve this?

  We could do the following: use a sphere for the head (since we can approximate
  it with a sphere if we make it tight), then use four pills for each of the
  limbs and one (or two, depending on the motions required) boxes for the torso.
  By using this method we can provide the granularity for intersections that we
  require for our game. Unfortunately, we have just introduced a new issue:
  instead of having to check against one bounding volume, we now have to check
  against 5 or 6 volumes. This can be easily solved by wrapping everything in
  another bounding volume (such as an AABB) that we can use to avoid checking
  against all the volumes until we have to.

  What we have just done here is create a hierarchy of bounding volumes in an
  effort to reduce the number of volumes to check against while still providing
  the granularity that we need. Notice that we incidentally also partitioned the
  space around the object (or parts of the model in our example). While this is
  a side-effect of a BVH, it does not make it a space partitioning algorithm.
  The key distinction is the following: when we construct a BVH, the volumes in
  which we divide the space may either overlap or contain one another, whereas a
  space partitioning structure will divide it into distinct and \emph{disjoint}
  sections. \\
  Another point to keep in mind is that the way we structured the hierarchy in
  our example is not necessarily the way all BVH are constructed. It is entirely
  possible for the parent volume to not enclose all of its children. While it is
  usually easier to have this property, it is by no means a requirement.

  The final point to discuss before we dive into the details of BVH is the
  following: who constructs the BVH? Should the artists/designers construct
  them?\\
  This question is a bit more subtle than it appears at first. While it may be
  tempting to delegate the construction of the BVH to the artist, we have to
  consider the fact that the hierarchy that they construct isn't necessarily
  optimized for intersection testing. It will most likely be made for easier
  editing and maintenance of the model itself. While this is not an unreasonable
  position for them, it is not what we would want for a collision detection
  system. The solution is then to automate as much as possible and construct the
  BVH at pre-processing time.

  \section{Hierarchy Design Issues}
    Before we can discuss how to create a BVH, traverse it, and intersect
    against it, we need to consider the following classical questions:
    \begin{itemize}
      \item What would we want from a BVH?
      \item What is the BVH going to cost us?
      \item Since this is a tree, what would the ideal degree be?
    \end{itemize}
